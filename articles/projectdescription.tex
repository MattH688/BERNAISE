\documentclass[a4paper,10pt]{article}
\usepackage{graphicx}
\usepackage{epstopdf}
\usepackage{epsfig}
\usepackage{amssymb,amsmath,stmaryrd,tabularx}
\usepackage{wrapfig}
\usepackage{bm}
\usepackage[center]{subfigure}
\usepackage{enumerate}
\usepackage{hyperref} %Not actually used. You can put multiple includegraphics commands right after each other without spaces without problems. This is only useful for separated captions.
%Prevent indentation of first line in paragraphs.
%\setlength{\parindent}{0pt}

\renewcommand{\v}[1]{\mathbf{#1}}
\newcommand{\intf}{\textrm{if}}

% Physics header, modified from dfdc.net %
\let\vaccent=\v % rename builtin command \v{} to \vaccent{}
\renewcommand{\v}[1]{\ensuremath{\mathbf{#1}}} % for vectors
\renewcommand{\times}{\cdot} % times symbol
\renewcommand{\vec}[1]{\v{#1}} % also for vectors
\newcommand{\gv}[1]{\ensuremath{\mbox{\boldmath$ #1 $}}} 
% for vectors of Greek letters
\newcommand{\uv}[1]{\ensuremath{\mathbf{\hat{#1}}}} % for unit vector
\newcommand{\abs}[1]{\left| #1 \right|} % for absolute value
\newcommand{\avg}[1]{\left< #1 \right>} % for average
\let\underdot=\d % rename builtin command \d{} to \underdot{}
\newcommand{\diff}{\mathrm{d}}
\renewcommand{\d}[2]{\frac{\diff #1}{\diff #2}} % for derivatives
\newcommand{\dd}[2]{\frac{\diff^2 #1}{\diff #2^2}} % for double derivatives
%\newcommand{\pd}[2]{\frac{\partial #1}{\partial #2}}
\newcommand{\pd}[2]{\frac{\partial #1}{\partial #2}}
\newcommand{\pdinl}[2]{\partial #1 / \partial #2 }
% for partial derivatives
\newcommand{\pdd}[2]{\frac{\partial^2 #1}{\partial #2^2}}
% for double partial derivatives
\newcommand{\pdc}[3]{\left( \frac{\partial #1}{\partial #2}
	\right)_{#3}} % for thermodynamic partial derivatives
\newcommand{\pdcinl}[3]{\left(\partial #1 / \partial #2 \right)_{#3}}
\newcommand{\ket}[1]{\left| #1 \right>} % for Dirac bras
\newcommand{\bra}[1]{\left< #1 \right|} % for Dirac kets
\newcommand{\braket}[2]{\left< #1 \vphantom{#2} \right|
	\left. #2 \vphantom{#1} \right>} % for Dirac brackets
\newcommand{\matrixel}[3]{\left< #1 \vphantom{#2#3} \right|
	#2 \left| #3 \vphantom{#1#2} \right>} % for Dirac matrix elements
\newcommand{\grad}[1]{\gv{\nabla} #1} % for gradient
\let\divsymb=\div % rename builtin command \div to \divsymb
\renewcommand{\div}[1]{\gv{\nabla} \cdot #1} % for divergence
\newcommand{\curl}[1]{\gv{\nabla} \times #1} % for curl
\newcommand{\mean}[1]{\left< #1 \right>}
\newcommand{\laplacian}[1]{\grad^2 #1}
\newcommand{\emphy}[1]{\textsc{#1}}
\newcommand{\pdt}[1]{\partial_t #1}
\newcommand{\sym}[1]{\mathrm{sym} #1 }
\newcommand{\inpr}[2]{\left< #1, #2 \right>}

\newcommand{\wall}{\mathrm{wall}}
\newcommand{\inlet}{\mathrm{in}}
\newcommand{\outlet}{\mathrm{out}}


\begin{document}
\section*{Project Description:\\Two-Phase Electrohydrodynamics in Porous Media}
\textbf{Participants:} Gaute Linga\footnote{Copenhagen University, Niels Bohr Institutet, Norwegian}, Asger J. S. Bolet\footnote{Copenhagen University, Niels Bohr Institutet, Danish}, plus possible one more if there is one interested.

\subsection*{Aim of the Project}
To simulate binary electrolyte dynamics in porous geometries, in order to studied effects of electro wetting and interface dynamics due to electric interactions which hopefully will allow us to study electro ``static'' intrusion in a capillary due surface charge effects and different Debye lengths in the two fluids, in particular in two dimensions.

\subsection*{Model}
The model we would like to solve is a binary fluid flow with dissolved ions and electric field applied via surface charges or forced potential difference, for that we have defined the following defused interface model, consisting of Navier--Stokes equation, Nernst--Planck equation, Gauss' law and a thermodynamically consistent interface tracking phase field.  
The full model consists of following set of equations:
The incompressible time-dependent Stokes equations for the fluid flow:
\begin{gather}
  \begin{split}
  \rho(\phi) \pdt \v v + \left( \left(\rho(\phi) \v v - \rho'(\phi) M(\phi) \grad g_\phi  \right) \cdot \grad \right) \v v - \div \left[  \mu(\phi)(\grad \v v + \grad \v v ^T) \right] + \grad p \\
  = - \rho_e \grad V - \tilde \sigma \xi\div \left[ \grad \phi \otimes \grad \phi \right] - \frac 1 2 \varepsilon'(\phi) | \grad V |^2 \grad \phi + \rho(\phi) \v g,
  \end{split}\\
  \div \v v = 0, \\
  \pdt \phi + \v v \cdot \grad \phi - \div(M(\phi) \grad g_\phi ) = 0, \\
  \pdt c_i + \v v \cdot \grad c_i - \div \left( D_{i} (\phi) \grad c_i + \frac{c_i q_e  z_i}{k_B T} \grad V +\frac{D_i(\phi)}{k_B T}\beta_i'(\phi) \grad \phi \right) = 0, \\
  \div ( \varepsilon (\phi) \grad V ) = - \rho_e.
\end{gather}
Here, $\rho$ is the density , $\v v$ is the velocity field, $\mu$ is the dynamic viscosity, $p$ is the pressure, $\tilde{\sigma}$ is the surface tension, $\xi$ is the interface width, $c_i$ is the concentration of the soluted species $i$, and the associated chemical potential have been taken as, $g_{c_i} = \alpha'(c_i)+\beta(\phi)$ where $\alpha(c_i) = k_B T c_i(\log c_i -1)$ and $beta_i$ is the ???, $D_i$ is the diffusion constant for the $i$ ion. Further $phi$ is the phase field which takes the value $phi = -1$ in phase $i=1$ and $phi = 1$ in phase $i=2$, $M(\phi)= \gamma (1 - \phi^2)^2$ is the mobility of the phase field, $\rho_e = \sum_i z_i c_i$ is the total charge density ($z_i$ is the valency of solute species $i$), and finally $V$ is the electric potential. All the quantities defined with a $Q(phi)$ is defined as: $Q(\phi) = 0.5(Q_1+Q_2)+0.5(Q_1-Q_2)\phi$ and $Q'(\phi) = 0.5(Q_1-Q_2)$ except if otherwise i stated, finally $g_{\phi}$ is given as:
\begin{align}
 g_\phi &= \tilde\sigma\xi^{-1}W'(\phi) - \tilde\sigma\xi\laplacian\phi + \sum_i \beta_i'(\phi) c_i - \frac 1 2 \varepsilon'(\phi) | \grad V |^2 .\\
 W(\phi) &= (1 - \phi^2)^2
\end{align} 
 

\subsection*{Methodolgy and challenges}
In order to simulate the above equations we will need to write down a variational from with a reasonable initialization and show it converges, especial the converges might pose a challenges as the system is far form linear. 
Further we will need to studied the systems many time scales by none dimensionalizing the equations, in order to integral whith a fitting time step. 
Lastly we will have to provide at good initial guess on the ionic system as the Gauss's Law and Nernst--Planck problem strongly coupled by any charged surface.

\subsection*{Expected outcome} 
The plan is to make a simulation environment for the above equations name BERNAISE---an acronym for Binary ElectRohydrodyNAmIc Simulation Environment.
The git repository can be found at \url{https://github.com/gautelinga/BERNAISE} along with an implementation plan called plan.
On the more physics oriented part we would like to make movies of intrusion of one of the phases in a 2-D porous sample after, some plots of reasonable converges.   


\end{document} 