\documentclass[a4paper,10pt]{article}
\usepackage{cite}
\usepackage[utf8]{inputenc}
\usepackage[T1]{fontenc}
\usepackage{graphicx}
\usepackage{amsmath, amssymb}
\usepackage{url}
\usepackage{hyperref}
\usepackage{multirow}
\numberwithin{equation}{section}
\numberwithin{figure}{section}
\usepackage{caption}
\usepackage{subcaption} %Not actually used. You can put multiple includegraphics commands right after each other without spaces without problems. This is only useful for separated captions.
%Prevent indentation of first line in paragraphs.
\setlength{\parindent}{0pt}


\begin{document}
\section{Project Description: Two Phase Electrohydrodynamics}
\textbf{Participants:} Gaute Linga\footnote{Copenhagen University, Niels Bohr Institutet, Norwegian}, Asger J. S. Bolet\footnote{Copenhagen University, Niels Bohr Institutet, Dane} plus possible one more if there is one interested.

\subsection{Description}
Our idea for a project in FEniCS is to implement a multi-physics solver for binary fluid flow with dissolved ions in them and surface charge interactions. The plan is to model this with the following set of equations:\newline 
The incompressible time dependent Stokes equation for the fluid flow:
\begin{align}
\rho(\psi) \partial_t \mathbf{u}  &=  \nabla p +\eta(\psi) \nabla^2 \mathbf{u} -  \rho_e  \nabla \phi + \mathbf{f}_{if}(\psi),  \\  
\nabla \cdot\mathbf{u} &= 0,
\end{align}
where $\rho$ is the fluid density, $\psi$ is the phase field that tracks the two phases,$u$ is the fluid velocity, $p$ is the pressure, $\eta$ is the dynamic viscosity,$\rho_e$ is the charge density, $\phi$ is the electric potential and  
$f_{if}$ is the interface force. \newline
Poissons equation for the electric potential: 
\begin{align}
\nabla\cdot\left(\epsilon_r(\psi)\nabla\phi\right) &= \frac{\rho_e}{\epsilon_0 },
\end{align}
where $\epsilon_0$ is the vacuum permittivity and $\epsilon_r$ the relative permittivity. \newline 
Nernst-Planck equation is to model the ion diffusion-convection-migration, one for each species of ion we will properly restrict our self to two:
\begin{align}
\partial_t c_i  &=  \nabla  \cdot  \left( - \mathbf{u} c_i  + D_i(\psi) \nabla c_i  +   \frac{D_i(\psi) z_i q_e}{k_B T} c_i \nabla \phi \right), \\
\rho_e &= \sum_i q_e z_i c_i, 
\end{align}
Where $c_i$ is the ion concentration, $D_i$ is the ion diffusivity, $z_i$ is the ion valency, $q_e$ is the electron charge, $k_B$ is Boltzmann's constant and $T$ is the temperature.\newline     
The Phase field equation for the tracking the interface and give the surface tension and possibly the contact angel however only for moderate angels: 
\begin{align}
\partial_t \psi + \mathbf{u}\cdot\nabla\psi &= M\nabla^2\left(\frac{3 \sigma }{W} \psi \left(\psi^2 -1\right) - \frac{3 \sigma W}{8}\nabla^2 \psi\right),
\end{align}
where $M$ is ???, $\sigma$ is ??? and $W$ is ???\newline 
What we would like to study with this system of equation is electric "static" intrusion in a capillary due surface charge effects and different Debye lengths in the two fluids.       
\subsection{Implementation in FEniCS}
In Order to simulate the above equations we will need to implement them in fenics. However there is some complications as system of equations is highly none-linear and have a lot off different time scales. On top of the a time integration schema we will have also have to slove the steady sated Poisson-Nernst-Planck-problem in order to give more realistic initial conditions. 
The simulation environment proposed for the above text have the name : BERNAISE an acronym for Binary ElectRohydrodyNAmIc Simulation Environment, the git repository can be found at https://github.com/gautelinga/BERNAISE along implementation plan called plan. 
 
   





\end{document} 