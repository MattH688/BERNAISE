\documentclass[a4paper,10pt]{article}
\usepackage{cite}
\usepackage[utf8]{inputenc}
\usepackage[T1]{fontenc}
\usepackage{graphicx}
\usepackage{amsmath, amssymb}
\usepackage{url}
\usepackage{hyperref}
\usepackage{multirow}
\numberwithin{equation}{section}
\numberwithin{figure}{section}
\usepackage{caption}
\usepackage{subcaption} %Not actually used. You can put multiple includegraphics commands right after each other without spaces without problems. This is only useful for separated captions.
%Prevent indentation of first line in paragraphs.
%\setlength{\parindent}{0pt}

\renewcommand{\v}[1]{\mathbf{#1}}
\newcommand{\intf}{\textrm{if}}

\begin{document}
\section*{Project Description:\\Two-Phase Electrohydrodynamics in Porous Media}
\textbf{Participants:} Gaute Linga\footnote{Copenhagen University, Niels Bohr Institutet, Norwegian}, Asger J. S. Bolet\footnote{Copenhagen University, Niels Bohr Institutet, Danish}, plus possible one more if there is one interested.

\subsection*{Description}
We will implement a multi-physics solver for binary fluid flow with dissolved ions in them and surface charge interactions.
The full model consists in the following set of equations:
\begin{itemize} 
\item The incompressible time-dependent Stokes equations for the fluid flow:
  \begin{align}
    \rho(\psi) \partial_t \mathbf{u}  &= \nabla p +\eta(\psi) \nabla^2 \mathbf{u} -  \rho_e  \nabla \phi + \mathbf{f}_{if}(\psi),  \\  
    \nabla \cdot\mathbf{u} &= 0,
  \end{align}
  where $\rho$ is the phase-dependent fluid density, $\psi$ is the phase field that tracks the interface between the two phases, $\v u$ is the fluid velocity, $p$ is the pressure, $\eta$ is the dynamic viscosity, $\rho_e$ is the charge density, $\phi$ is the electric potential and $\v f_{\intf}$ is the interface force.
\item Poisson's equation for the electric potential: 
  \begin{align}
    \nabla\cdot\left(\epsilon_r(\psi)\nabla\phi\right) &= \frac{\rho_e}{\epsilon_0 },
  \end{align}
  where $\epsilon_0$ is the vacuum  and $\epsilon_r$ the relative permittivity.
\item The phenomenological Nernst--Planck equation to model the ion diffusion-convection-migration, one for each species $i$ of ion (we will probably restrict ourselves to two):
  \begin{align}
    \partial_t c_i  &=  \nabla  \cdot  \left( - \mathbf{u} c_i  + D_i(\psi) \nabla c_i  +   \frac{D_i(\psi) z_i q_e}{k_B T} c_i \nabla \phi \right), \\
\rho_e &= \sum_i q_e z_i c_i, 
  \end{align}
  where $c_i$, $D_i$, $z_i$ are the ion concentration, diffusivity, and valency of species $i$, $q_e$ is the electron charge, $k_B$ is Boltzmann's constant and $T$ is the temperature.
\item The conservation equation for the phase field tracking the interface (which consequently gives the surface energy):
  \begin{align}
    \partial_t \psi + \mathbf{u}\cdot\nabla\psi &= M\nabla^2\left(\frac{3 \sigma }{W} \psi \left(\psi^2 -1\right) - \frac{3 \sigma W}{8}\nabla^2 \psi\right),
  \end{align}
  where $M$ is the mobility, $\sigma$ is the interface tension and $W$ is the interface width (which should approach 0 with grid refinement).
  The contact angle with the solid wall is implemented through the boundary conditions, and we will most likely implement a simple expression which is sufficient for moderate angles.
\end{itemize}
With this system of equations we will first study electric ``static'' intrusion in a capillary due surface charge effects and different Debye lengths in the two fluids, in particular in two dimensions.
Furthermore, we will study the effect of surface charge and a voltage bias upon drainage and imbibition in larger systems of synthetic porous media.

\subsection*{Implementation in FEniCS}
In order to simulate the above equations we will need to write down the variational forms and implement them in FEniCS.
However, there are some complications as the system of equations is highly non-linear and contains many different time scales.
On top of implementing a time integration scheme, we will have to solve the steady state Poisson--Nernst--Planck problem in order to provide realistic initial conditions.

The simulation environment proposed for the above text has the name BERNAISE---an acronym for Binary ElectRohydrodyNAmIc Simulation Environment.
The git repository can be found at \url{https://github.com/gautelinga/BERNAISE} along with an implementation plan called plan.

\end{document} 