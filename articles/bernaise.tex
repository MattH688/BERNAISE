\documentclass[reprint,prl,superscriptaddress]{revtex4-1}
\usepackage{graphicx}
\usepackage{epstopdf}
\usepackage{epsfig}
\usepackage{amssymb,amsmath,stmaryrd,tabularx}
\usepackage{wrapfig}
\usepackage{bm}
\usepackage[center]{subfigure}
\usepackage{enumerate}
\usepackage{cleveref}
\usepackage{hyperref}

% Physics header, modified from dfdc.net %
\let\vaccent=\v % rename builtin command \v{} to \vaccent{}
\renewcommand{\v}[1]{\ensuremath{\mathbf{#1}}} % for vectors
\renewcommand{\times}{\cdot} % times symbol
\renewcommand{\vec}[1]{\v{#1}} % also for vectors
\newcommand{\gv}[1]{\ensuremath{\mbox{\boldmath$ #1 $}}} 
% for vectors of Greek letters
\newcommand{\uv}[1]{\ensuremath{\mathbf{\hat{#1}}}} % for unit vector
\newcommand{\abs}[1]{\left| #1 \right|} % for absolute value
\newcommand{\avg}[1]{\left< #1 \right>} % for average
\let\underdot=\d % rename builtin command \d{} to \underdot{}
\newcommand{\diff}{\mathrm{d}}
\renewcommand{\d}[2]{\frac{\diff #1}{\diff #2}} % for derivatives
\newcommand{\dd}[2]{\frac{\diff^2 #1}{\diff #2^2}} % for double derivatives
%\newcommand{\pd}[2]{\frac{\partial #1}{\partial #2}}
\newcommand{\pd}[2]{\frac{\partial #1}{\partial #2}}
\newcommand{\pdinl}[2]{\partial #1 / \partial #2 }
% for partial derivatives
\newcommand{\pdd}[2]{\frac{\partial^2 #1}{\partial #2^2}}
% for double partial derivatives
\newcommand{\pdc}[3]{\left( \frac{\partial #1}{\partial #2}
 \right)_{#3}} % for thermodynamic partial derivatives
\newcommand{\pdcinl}[3]{\left(\partial #1 / \partial #2 \right)_{#3}}
\newcommand{\ket}[1]{\left| #1 \right>} % for Dirac bras
\newcommand{\bra}[1]{\left< #1 \right|} % for Dirac kets
\newcommand{\braket}[2]{\left< #1 \vphantom{#2} \right|
 \left. #2 \vphantom{#1} \right>} % for Dirac brackets
\newcommand{\matrixel}[3]{\left< #1 \vphantom{#2#3} \right|
 #2 \left| #3 \vphantom{#1#2} \right>} % for Dirac matrix elements
\newcommand{\grad}[1]{\gv{\nabla} #1} % for gradient
\let\divsymb=\div % rename builtin command \div to \divsymb
\renewcommand{\div}[1]{\gv{\nabla} \cdot #1} % for divergence
\newcommand{\curl}[1]{\gv{\nabla} \times #1} % for curl
\newcommand{\mean}[1]{\left< #1 \right>}
\newcommand{\laplacian}[1]{\grad^2 #1}
\newcommand{\emphy}[1]{\textsc{#1}}

\begin{document}
\title{Bernice: High-level/high-performance simulation of two-phase electrohydrodynamics in complex domains}

\newcommand{\NBI}{Niels Bohr Institute, University of Copenhagen, Blegdamsvej 17, DK-2200 Copenhagen, Denmark.}
\author{Asger J.~S.~Bolet}
\email{asger.bolet@nbi.dk}
\affiliation{\NBI}
\author{Gaute Linga}
\email{linga@nbi.dk}
\affiliation{\NBI}
\author{Joachim Mathiesen}
\email{mathies@nbi.dk}
\affiliation{\NBI}

% NOTE:
% Preferably, all sentences should be kept at one line only.
% This way it is easier to track changes in git.

\begin{abstract}
  \emph{Bernice} (\emphy{B}inary
  \emphy{E}lect\emphy{r}ohydrody\emphy{n}am\emphy{ic} Solv\emphy{e}r) is a high-level/high-performance finite element solver of two-phase electrohydrodynamic flow in complex geometries.
  The solver is implemented in Python via the FEniCS framework, which effectively utilizes the linear algebra backend PETSc, MPI and domain decomposition.
  The solver is validated by comparison to limiting cases where analytical solutions are available, and by the method of manifactured solution.
\end{abstract}

\maketitle

\section{Introduction}
Electric force.
This solver is inspired by the \emph{Oasis} solver for fluid flow \cite{mortensen2015}.

\bibliographystyle{apsrev4-1}
\bibliography{references}

\end{document}
