\documentclass[preprint,pre,superscriptaddress,a4paper]{revtex4-1}
\usepackage{graphicx}
\usepackage{epstopdf}
\usepackage{epsfig}
\usepackage{amssymb,amsmath,stmaryrd,tabularx}
\usepackage{wrapfig}
\usepackage{bm}
\usepackage[center]{subfigure}
\usepackage{enumerate}
\usepackage{hyperref}

% Physics header, modified from dfdc.net %
\let\vaccent=\v % rename builtin command \v{} to \vaccent{}
\renewcommand{\v}[1]{\ensuremath{\mathbf{#1}}} % for vectors
\renewcommand{\times}{\cdot} % times symbol
\renewcommand{\vec}[1]{\v{#1}} % also for vectors
\newcommand{\gv}[1]{\ensuremath{\mbox{\boldmath$ #1 $}}} 
% for vectors of Greek letters
\newcommand{\uv}[1]{\ensuremath{\mathbf{\hat{#1}}}} % for unit vector
\newcommand{\abs}[1]{\left| #1 \right|} % for absolute value
\newcommand{\avg}[1]{\left< #1 \right>} % for average
\let\underdot=\d % rename builtin command \d{} to \underdot{}
\newcommand{\diff}{\mathrm{d}}
\renewcommand{\d}[2]{\frac{\diff #1}{\diff #2}} % for derivatives
\newcommand{\dd}[2]{\frac{\diff^2 #1}{\diff #2^2}} % for double derivatives
%\newcommand{\pd}[2]{\frac{\partial #1}{\partial #2}}
\newcommand{\pd}[2]{\frac{\partial #1}{\partial #2}}
\newcommand{\pdinl}[2]{\partial #1 / \partial #2 }
% for partial derivatives
\newcommand{\pdd}[2]{\frac{\partial^2 #1}{\partial #2^2}}
% for double partial derivatives
\newcommand{\pdc}[3]{\left( \frac{\partial #1}{\partial #2}
 \right)_{#3}} % for thermodynamic partial derivatives
\newcommand{\pdcinl}[3]{\left(\partial #1 / \partial #2 \right)_{#3}}
\newcommand{\ket}[1]{\left| #1 \right>} % for Dirac bras
\newcommand{\bra}[1]{\left< #1 \right|} % for Dirac kets
\newcommand{\braket}[2]{\left< #1 \vphantom{#2} \right|
 \left. #2 \vphantom{#1} \right>} % for Dirac brackets
\newcommand{\matrixel}[3]{\left< #1 \vphantom{#2#3} \right|
 #2 \left| #3 \vphantom{#1#2} \right>} % for Dirac matrix elements
\newcommand{\grad}[1]{\gv{\nabla} #1} % for gradient
\let\divsymb=\div % rename builtin command \div to \divsymb
\renewcommand{\div}[1]{\gv{\nabla} \cdot #1} % for divergence
\newcommand{\curl}[1]{\gv{\nabla} \times #1} % for curl
\newcommand{\mean}[1]{\left< #1 \right>}
\newcommand{\laplacian}[1]{\grad^2 #1}
\newcommand{\emphy}[1]{\textsc{#1}}
\newcommand{\pdt}[1]{\partial_t #1}
\newcommand{\sym}[1]{\mathrm{sym} #1 }

\begin{document}
\title{Bernice: High-level/high-performance simulation of two-phase electrohydrodynamics in complex domains}

\newcommand{\NBI}{Niels Bohr Institute, University of Copenhagen, Blegdamsvej 17, DK-2200 Copenhagen, Denmark.}
\author{Asger J.~S.~Bolet}
\email{asger.bolet@nbi.dk}
\affiliation{\NBI}
\author{Gaute Linga}
\email{linga@nbi.dk}
\affiliation{\NBI}
\author{Joachim Mathiesen}
\email{mathies@nbi.dk}
\affiliation{\NBI}

% NOTE:
% All sentences should be kept at one line only.
% This way it is easier to track changes in git.

\begin{abstract}
  \emph{Bernice} (\emphy{B}inary
  \emphy{E}lect\emphy{r}ohydrody\emphy{n}am\emphy{ic} Solv\emphy{e}r) is a high-level/high-performance finite element solver of two-phase electrohydrodynamic flow in complex geometries.
  The solver is implemented in Python via the FEniCS framework, which effectively utilizes the linear algebra backend PETSc, MPI and domain decomposition.
  The solver is validated by comparison to limiting cases where analytical solutions are available, benchmark cases, and by the method of manifactured solution.
\end{abstract}

\maketitle

\section{Introduction}
This solver is inspired by the \emph{Oasis} solver for fluid flow \cite{mortensen2015}.

Applications of flow under the influence of electric fields: Micro-/millifluidics, MEMS, electrowetting-on-dielectric (EWOD), enhanced oil recovery using electric fields, dynamic capacitor.

Reviews of wetting phenomena \cite{degennes1985,bonn2009,snoeijer2013}.
Electrowetting \cite{mugele2005}.
Derivation of the equations of electrohydrodynamics (EHD) \cite{saville1997}.

Single-phase, steady-state EHD simulation in nanopores using FEniCS \cite{mitscha-baude2017}.
Simulation and comparison to experiments of electrowetting with contact line pinning \cite{walker2009}.
3D phase field modeling of electrohydrodynamic multiphase flows \cite{yang2013,yang2014}.
Electrohydrodynamics in microfluidic devices \cite{zeng2011}.

Review of diffuse interface models in fluid mechanics \cite{anderson1998}.
``Model H'' of \citet{hohenberg1977} (two incompressible fluids with matched densities and viscosities, Navier--Stokes--Cahn--Hilliard system, describes phase transitions of binary fluids or single-phase fluid near the critical point).
\citet{lowengrub1998} derived a thermodynamically consistent generalization of Model H where densities and viscosities were different in the two phases, however with the numerical difficulty that the velocity field was not divergency free.
Thermodynamically consistent and frame invariant phase field model for binary flow \cite{abels2012}.
Extension to EHD (i.e.~inclusion of electric fields and forces) by \citet{campillo-funollet2012}, where they also simulated the model using a dual mesh (see below).
A similar model for EHD with emphasis on contact line pinning was derived by \citet{nochetto2014}, but it does not appear to be frame-invariant as the chemical potential depends quadratically on velocity.

Numerical implementation of the model in \cite{abels2012} using a dual mesh formulation (FVM on the dual mesh for advection terms, FEM for the rest).
Benchmarks of bubble dynamics using phase-field models (without electrodynamics) \cite{aland2012} (based on \cite{hysing2009}).
Numerical schemes for the same case were presented and analyzed in \cite{guillen-gonzalez2014,grun2016}.
Finite element approximation of two-phase EHD (with Stokes flow assumption) \cite{nurnberg2017}.
Implementation of the full EHD model in \cite{campillo-funollet2012} by \cite{metzger2015}.

Critical review of wetting boundary conditions for phase-field models \cite{huang2015}.

\section{Model}
The governing equations of two-phase EHD can be summarized as the coupled system of two-phase flow, passive chemical transport, and electrostatics \cite{campillo-funollet2012}.
In order to track the interface between the phases, we introduce an order parameter field $\phi$ which attains the values $\pm 1$ respectively in the two phases, and interpolates between the two across the (diffuse) interface of thickness $\epsilon$.
In the sharp-interface limit $\epsilon \to 0$, the equations should reproduce the correct physics.

\subsection{Sharp-interface equations}
In the following, we present each equation of the physical (sharp-interface) model.
Down to the nanometer scale, the fluid flow is described by the incompressible Navier--Stokes equations (with some additional force terms):
\begin{gather}
  \rho_i \left( \pdt {\v v} + (\v v \cdot \grad ) \v v \right)  - \mu_i \laplacian \v v + \grad p = \sum_j g_{c_j} \grad c_j - \rho_e \grad V, \label{eq:sharp_NS1}\\
  \div \v v = 0. \label{eq:sharp_NS2}
\end{gather}
Here, $\rho_i$ is the density of phase $i$, $\v v$ is the velocity field, $\mu_i$ is the dynamic viscosity of phase $i$, $p(\v x, t)$ is the pressure field., $c_j (\v x, t)$ is the concentration of solute species $j$, and $g_{c_j}$ is the associated chemical potential, $ \rho_e = \sum_j z_j c_j$ is the total charge density ($z_j$ is the charge if solute species $j$), and finally $V$ is the electric potential.

The transport of the concentration field of species $i$ is governed by the conservative (advection--diffusion) equation
\begin{equation}
  \pdt c_i + \v v \cdot \grad c_i - \div (K_{j} c_i \grad (g_{c_i} + z_i V)) = 0,
  \label{eq:sharp_conc}
\end{equation}
where $K_{j}$ is the mobility in phase $j$.
With an appropriate choice of chemical potential $g_{c_i}$, the equations reduce to the phenomenological Nernst--Planck equation which has been established for the motion of charged species in solution under influence of an electric field. The Nernst--Planck equation reads
\begin{equation}
  \pdt c_i + \v v \cdot \grad c_i + \div \left[ D_{ij} \grad c_i + \frac{D_{ij} z_i}{k_{\rm B} T} c_i \grad V \right] = 0,
  \label{eq:nernst-planck}
\end{equation}
which coincides with Eq.~\eqref{eq:sharp_conc} with the choices $g_{c_i} = k_{\rm B} T \log c_i + \textrm{const.}$, $K_{ij} = D_{ij}/(k_{\rm B} T)$.
This amounts to a weak/dilute solution.

Since the dynamics of the electric field is much faster than the dynamics of charge transport, we can safely assume electrostatic conditions.
This amounts to solving the Poisson problem (Gauss' law)
\begin{equation}
  \div ( \varepsilon_i \grad V ) = - \rho_e,
\end{equation}
where $\varepsilon_i = \varepsilon_0 \varepsilon_{\textrm{r}, i}$ is the permittivity of phase $i$, where, again, $\varepsilon_0$ is the vacuum permitivity and $\varepsilon_{\textrm{r}, i}$ is the relative permeability of phase $i$. 

Don't forget the boundary conditions!

\subsection{Phase-field formulation}
A thermodynamically consistent phase-field model which reduces to this formulation was proposed by \cite{campillo-funollet2012}:
\begin{gather}
  \begin{split}
    \pdt (\rho (\phi) \v v) + \div \left( \rho(\phi) \v v \otimes \v v \right) - \div \left[ \mu (\phi) (\grad \v v + \grad \v v ^T) + \v v \otimes \rho'(\phi) M(\phi) \grad g_\phi \right] + \grad p\\
    = g_\phi \grad \phi + \sum_i g_{c_i} \grad c_i - \rho_e \grad V + \rho(\phi) \v g,
  \end{split} \label{eq:PF_NS1}\\
  \div \v v = 0, \\
  \pdt \phi + \v v \cdot \grad \phi - \div(M(\phi) \grad g_\phi ) = 0, \label{eq:PF1}\\
  \pdt c_i + \v v \cdot \grad c_i - \div ( K_i (\phi) c_i \grad (g_{c_i} + z_i V )) = 0, \\
  \grad ( \varepsilon (\phi) \grad V ) = - \rho_e.
\end{gather}
Here, $\phi$ is the phase field, and it takes the value $\phi=-1$ in phase $i=1$, and the value $\phi=1$ in phase $i=2$.
Eq.~\eqref{eq:PF1} governs the conservative evolution of the phase field.
Here, $\rho$, $\mu$, $\varepsilon$, $K_i$ depend on which phase they are in and are considered slave variables of the phase field $\phi$.
At the interface they interpolate between the values in the two phases:
\begin{align}
  \rho (\phi) &= \frac{\rho_1 + \rho_2}{2} + \frac{\rho_1 - \rho_2}{2} \phi, \\
  \mu (\phi ) &= \frac{\mu_1 + \mu_2}{2} + \frac{\mu_1 - \mu_2}{2} \phi, \\
  \varepsilon (\phi) &= \frac{\varepsilon_1 + \varepsilon_2}{2} + \frac{\varepsilon_1 - \varepsilon_2}{2} \phi, \\
  K_i (\phi) &= \frac{K_{i,1} + K_{i,2}}{2} + \frac{K_{i,1} - K_{i,2}}{2} \phi.
\end{align}
Moreover, the chemical potential of species $c_i$ is given by
\begin{align}
  g_{c_i}(c_i, \phi) = \alpha'(c_i) + \beta_i (\phi),
  \label{eq:PF_chempot}
\end{align}
where, consistently with the Nernst--Planck equation, we model $\alpha(c) = k_{\mathrm{B}} T c (\log c - 1) $.
The ``chemical potential'' of the phase field is defined as
\begin{equation}
  % g_\phi = \frac{3 \sigma}{2 \sqrt{2}} \left[ \epsilon^{-1} W'(\phi) - \epsilon \laplacian \phi \right] + ,
  g_\phi = \pd f \phi - \div \pd f {\grad \phi} + \sum_i \beta_i ' (\phi) c_i - \frac 1 2 \varepsilon'(\phi) | \grad V |^2 .
\end{equation}
The free energy functional $f$ of the phase field is defined by
\begin{equation}
  f(\phi, \grad \phi) = \frac{3 \sigma}{2 \sqrt{2}} \left[ \frac{\epsilon}{2} |\grad \phi|^2 + \epsilon^{-1} W(\phi) \right] = \tilde \sigma \left[ \frac{\epsilon}{2} |\grad \phi|^2 + \epsilon^{-1} W(\phi) \right],
\end{equation}
where where $\sigma$ is the surface tension, $\epsilon$ is the interface thickness, and $W(\phi)$ is a double well potential.
Here, we use, unless otherwise stated, $W(\phi) = (1-\phi^2)^2/4$.

Due to the form of Eq.~\eqref{eq:PF_chempot}, we can write
\begin{equation}
  g_{c_i} \grad c_i = \alpha'(c_i )\grad c_i + \beta_i (\phi) \grad c_i = \grad ( \alpha(c_i) + \beta_i(\phi) c_i) - \beta_i'(\phi) c_i \grad \phi,
\end{equation}
and thereby,
\begin{align}
  g_\phi \grad \phi + \sum_i g_{c_i} \grad c_i &= \left( \pd f \phi - \div \pd f {\grad \phi} - \frac 1 2 \varepsilon'(\phi) | \grad V |^2\right) \grad \phi + \grad ( \alpha(c_i) + \beta_i(\phi) c_i) \\
                                               \begin{split} = \grad \left( f(\phi, \grad \phi) + \sum_i ( \alpha(c_i) + \beta_i(\phi) c_i) \right) - \div \left[ \pd f {\grad \phi} \otimes \grad \phi \right] \\ - \frac 1 2 \grad \varepsilon(\phi) | \grad V |^2.
                                                 \end{split}
\end{align}
The contribution from the first term can be absorbed into the pressure by the (re)definition
\begin{equation}
  \tilde p = p - f(\phi, \grad \phi) - \sum_i ( \alpha(c_i) + \beta_i(\phi) c_i ) .
\end{equation}
The second and third terms are interface forces.
After some rewriting (assuming $\rho(\phi)$ is constant), Eq.~\eqref{eq:PF_NS1} can be expressed as
\begin{equation}
  \begin{split}
    \rho(\phi) \pdt \v v + \left( \left(\rho(\phi) \v v - \rho'(\phi) M(\phi) \grad g_\phi  \right) \cdot \grad \right) \v v -
    \div \left[ 2 \mu(\phi) \sym (\grad \v v) \right] + \grad p \\
    = - \rho_e \grad V - \div \left[ \pd f {\grad \phi} \otimes \grad \phi \right]  - \frac 1 2 \varepsilon'(\phi) | \grad V |^2 \grad \phi + \rho(\phi) \v g
  \end{split}
\end{equation}
where we have omitted the tilde over $p$. Inserting for $f$, we obtain
\begin{equation}
  \begin{split}
    \rho(\phi) \pdt \v v + \left( \left(\rho(\phi) \v v - \rho'(\phi) M(\phi) \grad g_\phi  \right) \cdot \grad \right) \v v -
    \div \left[ 2 \mu(\phi) \sym (\grad \v v) \right] + \grad p \\
    = - \rho_e \grad V - \tilde \sigma \epsilon\div \left[ \grad \phi \otimes \grad \phi \right]  - \frac 1 2 \varepsilon'(\phi) | \grad V |^2 \grad \phi + \rho(\phi) \v g .
  \end{split}
\end{equation}

\subsection{Nondimensionalization}

\section{Discretization}
The whole system of equations could in principle be solved simultaneously using implicit Euler discretization in time and e.g.~Newton's method to solve the nonliner system.
However, in order to simulate larger systems faster, it is preferable to use a splitting scheme to solve for each field sequentially.
One such splitting scheme was proposed in \cite{metzger2015}, based on the scheme without electrolytes as developed by \cite{guillen-gonzalez2014,grun2016}.
We will here discuss various splitting strategies.

\subsection{A splitting scheme}
A general splitting scheme is outlined here.
\begin{enumerate}
\item Propagate the phase field $\phi$. \label{pt:splitscheme_phi}
\item Propagate the chemical species concentrations $c_i$. \label{pt:splitscheme_ci}
\item Update the electric potential $V$ \label{pt:splitscheme_V}
\item Propagate the velocity $\v v$ and pressure $p$. \label{pt:splitscheme_vp}
\end{enumerate}
To simplify the complexity of the problem, the scheme can be reduced to describe reduced settings containing less physics.
The simplest is \emph{pure two-phase flow} (with unmatched densities and viscosities), where points \ref{pt:splitscheme_phi} and \ref{pt:splitscheme_vp} from the list above enter.
In the absence of electric charges and external electric potential, two-phase flow with passive transport of species can be modelled, i.e.~using the points \ref{pt:splitscheme_phi}, \ref{pt:splitscheme_ci} and \ref{pt:splitscheme_vp}.
Finally, the full two-phase EHD includes all the above points.
Testing the solver should therefore follow these three steps of complexity.

\bibliographystyle{apsrev4-1}
\bibliography{references}

\end{document}
